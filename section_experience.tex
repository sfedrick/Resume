% CC BY-SA 4.0 (https://creativecommons.org/licenses/by-sa/4.0/)
%Section: Work Experience at the top
\sectionTitle{Professional Experience}{\faSuitcase}
%\renewcommand{\labelitemi}{$\bullet$}
\begin{experiences}

\experience
  {current}      {Mechatronics Engineer |  Motion Planning and Controls }{ASML}{Wilton,Ct}
  {Jan 2022} {Currently I work to implement robotic (Controls, Machine learning, Kinematics, and Computer vision) software and hardware changes onto a 6 dof, precision, robotic reticle stage. This involves writing software in matlab and python for a 6 dof robotic reticle stage to simulate the new feature. I then implement the function onto a fleet of stages within a cross sectoral team. 
			\begin{itemize}
			\item Designed, developed, and tested  a software algorithm  that used neural networks and optimization techniques to optimize motor control 					parameters for reluctance actuators given robotic force data. Data was collected using a HIL (hardware in the loop) approach. 
			\item Refactored a code base designed to simulate the dynamics of a 6 dof robotic reticle stage. Transfered this code base into a modern version 					control system (git) from svn to aid software collaboration and tracking.
			\item Designed, developed, and tested a software algorithm that used the kinematics of a robotic stage to prevent thermal stresses 						from damaging the motors. 
           		\item Created a motion planning application in python that abstracted away positioning the 6 dof reticle stage into several desired positions.
           		\item Lead the development of two new mechatronic tools that will enable automatic testing of electro-
nics,hardware, and control methodologies of a 6 dof robotic reticle stage.
		 	\end{itemize} 
                 }
                {Robotics, controls, Machine Learning,Data Analysis, Tensor flow, Pytorch, Scipy,Simulation, Python, C++, Linear Algebra, Statistics, Agile}   
 \experience
    {Dec 2021}     {GRASP Lab | Graduate Student Researcher (Robotics)  }{ University of Pennsylvania}{Pennsylvania}
    {May 2021}    {I used a phase change material coupled with a heated insert to create a latching mechanism to add directionality to an origami robot. I then designed and implemented a controller in C++ that ran on a micro controller in real time to control the mechanism .
                      \begin{itemize}
			\item Designed and optimized a nonlinear controller using ILC (Iterative Learning Control) 
			\item wrote a monte carlo application in python to pick design parameters for an origami robot.
			\item Wrote a controller in C++ to control the latching mechanism.
			\item  See DOI:10.1109/ICRA40945.2020.9196534 for more information on the robot.
                      \end{itemize}
                    }
                    {C++, Python,Controls, Rapid Prototyping,Git,Docker, Data Analysis, Robotics}
  \experience
    {May 2020}   {Fluid dynamic research | Student Researcher}{Haverford College and University of Pennsylvania}{Pennsylvania}
    {December 2018} {
  I worked in collaboration with University of Pennsylvania and Haverford College to investigate the way Non-Newtonian effects impacted lubrication forces within a fluid.
                      \begin{itemize}
                        \item  Used Python to perform data analysis on real word data to determine the dynamics of a complex non-newtonian fluid system.
                         \item Analyzed and tracked mechanics of a sphere moving through a fluid using OpenCV.
                      \end{itemize}
                    }
                    {OpenCV, Python, Computational Physics, Computer Vision, Rapid Prototyping}
  \experience
    {May 2019}   {Digital Scholarship | Website designer}{Haverford College}{Pennsylvania}
    {December 2016} {
    I worked on https://archivogam.haverford.edu/en/, a website designed to connect persons illegally detained and forcibly disappeared in Guatemala during the Civil War with friends and relatives.
                      \begin{itemize}
                       \item Wrote the front and back end of Home and Images Section of Archivo Gam using django.
		     \item Implemented a panning zoom feature and a person search feature.
                      \end{itemize}
                    }
                    {Python, Linux, Django, git, command line }
\end{experiences}
