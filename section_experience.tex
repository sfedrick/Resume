% CC BY-SA 4.0 (https://creativecommons.org/licenses/by-sa/4.0/)
%Section: Work Experience at the top
\sectionTitle{Professional Experience}{\faSuitcase}
%\renewcommand{\labelitemi}{$\bullet$}
\begin{experiences}

\experience
  {current}      {Mechatronics Design Engineer |  Motion Planning and Controls }{ASML}{Wilton,Ct}
  {Jan 2022} {
                  Currently I work to implement robotic (Controls, Machine learning, Kinematics, and Computer vision) software and functional changes onto a 6 dof, precision, reticle stage. This involves writing software in matlab, python and C++ for a 6dof reticle stage. I then implement the function onto a fleet of stages within a cross sectoral team. 
			\begin{itemize}
		\item Designed, developed, and tested  a software algorithm in matlab to optimize pretension values for pull only actuators and to optimize controller parameters using a MIL (machine in the loop) approach. The algorithm was then implemented in C++ to function on a 6 dof reticle stage.
	\item Refactored a code base designed to simulate the dynamics of the reticle stage. Transfered this code base into a modern version control system (git) to aid software collaboration along with adding a new feature to said code base.
                       \item Created a motion planning application in python that utilized multithreading to position the 6 dof reticle stage into several desired positions  .
		    \item Lead the development of two new mechatronic tools that will enable automatic testing of electronics,hardware, and control methodologies of our reticle stage.
		 \item Designed, developed, and tested a software algorithm in python to keep motor coils warm while preventing movement of the robot in order to minimize thermal stresses due to bonding layers undergoing a phase transition as the motor coils cooled under no load. The algorithm was then implemented in C++ to function on a 6dof reticle stage.
		
 			\end{itemize} 
                 }
                {robotics, controls, Data Analysis, Simulation, Matlab, Python, C++, Linear Algebra, Statistics, Agile}   
 \experience
    {Dec 2021}     {GRASP Lab | Graduate Student Researcher (Robotics)  }{ University of Pennsylvania}{Pennsylvania}
    {May 2021}    {
 I used a phase change material coupled with a heated insert to create a latching mechanism to add directionality to an origami robot. I then designed and implemented a controller in C++ that ran on a micro controller in real time to control the mechanism .
                      \begin{itemize}
			\item Designed and optimized a nonlinear controller using Matlab and Python 
			\item Created a simulation and optimization of a mechanical Design  of the latch insert using Python.
			\item Wrote a controller in C++ to control the latching mechanism.
			\item  See DOI:10.1109/ICRA40945.2020.9196534 for more information on the robot.
                      \end{itemize}
                    }
                    {C++, Python,Controls, Rapid Prototyping,Git,Docker, Data Analysis, Robotics}
  \experience
    {May 2020}   {Fluid dynamic research | Student Researcher}{Haverford College and University of Pennsylvania}{Pennsylvania}
    {December 2018} {
  I worked in collaboration with University of Pennsylvania and Haverford College to investigate the way Non-Newtonian effects impacted lubrication forces within a fluid.
                      \begin{itemize}
                        \item Analyzed and tracked mechanics of a sphere moving through a fluid using OpenCV 
                      \end{itemize}
                    }
                    {Matlab, OpenCV, Python, Solid Works, Java, Computational Physics, Computer Vision, Rapid Prototyping}
  \experience
    {May 2019}   {Digital Scholarship | Website designer}{Haverford College}{Pennsylvania}
    {December 2016} {
    I worked on https://archivogam.haverford.edu/en/, a website designed to connect persons illegally detained and forcibly disappeared in Guatemala during the Civil War with friends and relatives.
                      \begin{itemize}
                       \item Wrote the front and back end of Home and Images Section of Archivo Gam 
		     \item Implemented a panning zoom feature and a person search feature
                      \end{itemize}
                    }
                    {Python, Linux, Django, git, command line ,Docker}

 
 

\end{experiences}
